\thispagestyle{empty}

\vspace*{10ex}

\lskopf

%\vskip -1ex \hrule height0pt depth0pt

\sloppy
\begin{center}
  {\large\bf Klausur zur Vorlesung \\
    {\it \vorlesungstitel}\\[0.3cm]
             \semester}\\[0.3cm]
  {\bf \klausurdatum}\\[1ex]
\end{center}

  \vspace*{2cm}

\begin{center}


\end{center}

\noindent Hiermit bestätige ich, dass ich vor Prüfungsbeginn darüber in Kenntnis gesetzt wurde, dass ich
im Falle einer plötzlich während der Prüfung auftretenden Erkrankung das Aufsichtspersonal umgehend
informieren muss. Dies wird im Prüfungsprotokoll vermerkt. Danach muss unverzüglich ein Rücktritt von
der Prüfung beim zuständigen Prüfungsausschuss beantragt werden. Ein vertrauensärztliches Attest --
ausgestellt am Prüfungstag -- kann gegebenenfalls innerhalb der nächsten Tage nachgereicht werden. Wird
die Prüfung hingegen in Kenntnis der gesundheitlichen Beeinträchtigung dennoch regulär beendet, kann im
Nachhinein kein Prüfungsrücktritt aufgrund von Krankheit beantragt werden.

\vspace{1.3cm}

  \begin{tabular}{r c}
    Garching, den \klausurdatum & \rule{5.5cm}{.3pt} \\
    & (Unterschrift)
  \end{tabular}\\


  \newpage
  \thispagestyle{empty}
\subsection*{Wichtige Hinweise zur Klausur:}
\begin{description}
  \item [Bearbeitungszeit] \
     90 Minuten
  \item [Unterlagen] \
  \begin{itemize}
    \item Verwenden Sie zur Bearbeitung der Aufgaben bitte nur die ausgeteilten Blätter.
    \item Beschriften Sie jedes Blatt mit Ihrem Namen und Ihrer Matrikelnummer!
    \item Alle Blätter (Angaben, bearbeitete Aufgaben, Blätter mit Notizen) müssen abgegeben werden.
    \item Kontrollieren Sie die Vollständigkeit Ihrer Unterlagen. Die Klausur umfasst
    \begin{itemize}
      \item \textbf{\AnzahlSeiten\ Seiten bzw.\ \AnzahlBlaetter\ Blätter}  (inklusive des Deckblatts)
      \item \textbf{\AnzahlAufgaben\ Aufgaben}
    \end{itemize}
    Sollten Sie feststellen, dass Ihre Unterlagen nicht vollständig sind, weisen Sie bitte umgehend die
    Klausuraufsicht darauf hin!
  \end{itemize}

  \item [Klausurergebnis] \
  \begin{itemize}
    % \item Beschriften Sie den beiliegenden Briefumschlag mit Ihrer Adresse, damit Ihnen
    % das Klausurergebnis per Post zugestellt werden kann.
    \item
    Die Klausurergebnisse werden auch über das TUMonline-Portal bekannt gegeben.
    \item Der Termin zur Einsichtnahme wird \klausureinsicht\ im Internet
      auf den Übungsseiten bekannt gegeben.
  \end{itemize}

  \item [Aufgaben] \
  \begin{itemize}
    \item  Verwenden Sie bitte \textbf{keine} Bleistifte oder rote oder grüne Stifte.
    
    \item Verwenden Sie falls nicht anders angegeben die Algorithmen, die in Vorlesung und Übung verwendet
    wurden. Selbstentwickelte Algorithmen führen im Allgemeinen zu Punktabzug.
    
    \item Falls nicht anders angegeben, so lösen Sie Aufgaben zum Thema SQL grundsätzlich nur mit den Mitteln des SQL-92 Standards und keinesfalls mit nicht standartisierten Konstrukten.

    \item Einige Aufgaben sind so formuliert, dass Sie die vorgegebenen Antworten
    als richtig oder falsch bewerten müssen. Bei Fragestellungen mit zwei Spalten (für richtig und falsch)
    ist je Zeile nur eine der beiden Optionen auszuwählen.

    \textbf{Korrekt angekreuzte Felder geben Punkte, falsch angekreuzte Felder führen zu Punktabzug.
    Leergelassene Felder ändern die Punktzahl nicht. Die Gesamtpunktzahl einer Aufgabe kann jedoch nicht
    unter 0 Punkte sinken.}

    \textbf{Kreuze müssen eindeutig zuzuordnen sein,} d.h.\ Kreuze die mehrere Spalten/Zeilen überdecken
    werden nicht gewertet!


  \end{itemize}
\end{description}
\newpage
